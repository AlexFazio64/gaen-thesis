\chapter{Introduzione}
\label{ch:introduzione}

\section{Contesto}
\label{sec:contesto}
Nell'era digitale, l'accesso immediato a contenuti
multimediali ha trasformato radicalmente il modo in cui
fruiamo di film, serie TV, libri e videogiochi.
Tuttavia, questa abbondanza di informazioni porta con sé
una sfida: la crescente esposizione a spoiler.
Gli spoiler, rivelando anticipatamente elementi cruciali
della trama, possono compromettere l'esperienza di
fruizione, generando frustrazione e delusione negli utenti.

Il problema degli spoiler è particolarmente rilevante in
contesti online, dove le discussioni e le recensioni
proliferano su forum, social media e piattaforme di
streaming.
La mancanza di strumenti efficaci per il rilevamento
automatico di spoiler rende difficile per gli utenti
proteggersi da tali rivelazioni indesiderate.

Questa tesi si propone di esplorare l'utilizzo di Large
Language Models (LLM) per affrontare la sfida del
rilevamento di spoiler.
L'obiettivo è sviluppare un sistema in grado di
identificare automaticamente gli spoiler in testi di varia
natura, sfruttando le capacità di comprensione del
linguaggio naturale degli LLM.

\section{Motivazioni}
\label{sec:motivazioni}
La motivazione principale di questa ricerca risiede nella
volontà di contribuire allo sviluppo di strumenti che
migliorino l'esperienza di fruizione dei contenuti
multimediali.
A tale scopo, è stato realizzato un sistema di rilevamento
di spoiler integrato in una estensione per browser, che
offre agli utenti un controllo maggiore sulla propria
esposizione agli spoiler.

Questa ricerca intende esplorare il potenziale degli LLM in
un'applicazione pratica e rilevante, provando a contribuire
alla comprensione delle loro capacità e limitazioni nel
contesto del rilevamento di informazioni specifiche in
testi complessi.

\section{Obiettivi}
\label{sec:obiettivi}

Gli obiettivi di questa tesi sono i seguenti:

\begin{itemize}
  \item Realizzare un sistema di rilevamento di spoiler
        basato su Large Language Models.
  \item Valutare l'efficacia del sistema sviluppato nel
        rilevamento di spoiler in testi di varia natura.
  \item Integrare il sistema in un'estensione per browser
        e valutarne l'utilità e l'usabilità.
\end{itemize}

\section{Struttura della tesi}
\label{sec:struttura-tesi}

Il resto della tesi è organizzato come segue:

\begin{itemize}
  \item Nel Capitolo~\ref{ch:background} vengono
        presentati i concetti e le tecnologie alla base
        della ricerca, con particolare attenzione ai Large
        Language Models.
  \item Nel Capitolo~\ref{ch:lavori-correlati} vengono
        esaminati i lavori correlati, con un focus sulle
        ricerche relative al rilevamento di spoiler.
  \item Nel Capitolo~\ref{ch:sistema} viene descritto il
        sistema di rilevamento di spoiler sviluppato, con
        particolare attenzione alla progettazione e
        all'implementazione.
  \item Nel Capitolo~\ref{ch:valutazione} vengono
        presentati i risultati dell'analisi sperimentale
        condotta per valutare l'efficacia del sistema.
  \item Nel Capitolo~\ref{ch:conclusioni} vengono
        riassunti i risultati ottenuti e vengono discusse
        le possibili direzioni future di ricerca.
\end{itemize}

