\chapter{Background}
\label{ch:background}

\section{Approcci al problema}
\label{sec:approcci}

Di seguito vengono presentati alcuni approcci al problema
di rilevazione di spoiler in un testo.

\subsection{Approccio basato su regole}
\label{subsec:approccio-regole}

Un approccio molto semplice consiste nell'identificare
alcune parole chiave che sono tipicamente presenti in un
testo che contiene spoiler.
Ad esempio, le parole ``spoiler'' e ``spoiler alert'' sono
spesso utilizzate per avvertire il lettore che il testo
successivo contiene informazioni che potrebbero rovinare la
visione di un film o la lettura di un libro.
Altri esempi di parole chiave potrebbero essere i nomi di
personaggi o luoghi chiave della trama.
Questo approccio è molto semplice e può essere implementato
con poche righe di codice, ma ha il difetto di essere molto
limitato e di non essere in grado di rilevare spoiler più
sottili o nascosti.

Questo approccio è inoltre soggetto a molti falsi positivi,
ovvero a casi in cui il testo viene erroneamente
classificato come contenente spoiler quando in realtà non
lo contiene.
Ad esempio, una recensione di un film potrebbe contenere il
nome di un personaggio chiave della trama senza rivelare
informazioni cruciali sulla storia.
In questo caso, il testo non dovrebbe essere classificato
come contenente spoiler, ma l'approccio basato su regole
potrebbe erroneamente classificarlo come contenente
spoiler.

Oltre ai falsi positivi, questo approccio è anche soggetto
a falsi negativi, ovvero a casi in cui il testo contiene
spoiler ma non viene classificato come tale.
Ad esempio, una recensione di un film potrebbe contenere
informazioni cruciali sulla trama senza utilizzare le
parole chiave tipicamente associate agli spoiler, o in casi
anche più semplificati, potrebbe essere scritta in una
lingua diversa da quella in cui sono state definite le
parole chiave o contenere errori di battitura e di
ortografia che impediscono al sistema di riconoscere le
parole chiave.

Per questi motivi, l'approccio basato su regole non
promette risultati soddisfacenti e si è deciso di non
adottarlo in questo lavoro.

\subsection{Approccio basato sul machine learning}
\label{subsec:approccio-ml}

Un approccio più sofisticato e promettente è quello basato
sul machine learning.
In questo approccio, si addestra un modello di machine
learning su un insieme di dati di addestramento
etichettati, ovvero un insieme di testi già classificati
come contenenti spoiler o non contenenti spoiler.
Il modello di machine learning impara a riconoscere i
pattern nei dati di addestramento e a classificare i testi
in base a tali pattern.
Una volta addestrato, il modello può essere utilizzato per
classificare nuovi testi come contenenti spoiler o non
contenenti spoiler.

Questo approccio ha il vantaggio di essere molto più
flessibile e potente rispetto all'approccio basato su
regole, in quanto il modello di machine learning è in grado
di riconoscere pattern complessi e sottili nei dati e di
generalizzare tali pattern ai nuovi dati.
Inoltre, il modello di machine learning è in grado di
adattarsi autonomamente ai nuovi dati, senza la necessità
di modificare manualmente le regole o i parametri del
sistema.

Tuttavia, l'approccio basato sul machine learning ha anche
alcuni svantaggi.
In primo luogo, richiede un insieme di dati di
addestramento etichettati, che possono essere costosi e
laboriosi da ottenere a seconda del dominio di
applicazione.

Il secondo svantaggio è che il modello di machine learning
è difficile da realizzare e richiede competenze avanzate
per la sua implementazione.
Inoltre, il modello deve essere addestrato su un insieme di
dati di addestramento rappresentativo e bilanciato,
altrimenti potrebbe essere soggetto a \textit{overfitting}
o a \textit{underfitting}.

Generalizzando, l'approccio basato sul machine learning è
più complesso e richiede più risorse rispetto all'approccio
basato su regole, ma promette risultati migliori e più
accurati.

\section{Approccio proposto}
\label{sec:approccio-proposto}

Per affrontare il problema della rilevazione di spoiler in
un testo, si propone di utilizzare un approccio basato sul
machine learning.

L'intuizione alla base di questo approccio è che i LLM
(Large Language Model) sono in grado di catturare le
relazioni semantiche e sintattiche tra le parole e i
concetti all'interno di un testo, e quindi di riconoscere i
pattern che caratterizzano i testi contenenti spoiler.

In particolare, si propone di utilizzare un modello di LLM
pre-addestrato con lo scopo di \textbf{estrarre} lo spoiler
da un testo piuttosto che classificare l'intero testo come
contenente spoiler o non contenente spoiler, nel caso in
cui il testo contenga spoiler.

Utilizzare un modello di LLM pre-addestrato ha il vantaggio
di non richiedere un insieme di dati di addestramento
etichettati, in quanto il modello è già stato addestrato su
un vasto insieme di dati e ha imparato a riconoscere i
pattern nei testi in modo automatico e generale.

A prescindere dal modello di LLM utilizzato, si propone di
utilizzare un approccio basato su \textbf{RAG}
(\textit{Retrieval-Augmented
  Generation})\cite{lewis2020retrievalaugmented} per
arricchire il testo con informazioni realitive al contesto
in cui è stato scritto.
Ciò dovrebbe migliorare la qualità delle predizioni del
modello di LLM, in quanto il contesto può influenzare il
significato delle parole e dei concetti all'interno del
testo.

La difficolta principale di questo approccio è che i
modelli di LLM sono molto complessi e richiedono risorse
computazionali e di memoria considerevoli per essere
addestrati e utilizzati.
Inoltre, i modelli di LLM sono soggetti a fenomeni di
allucinazione e di bias, che possono influenzare le
predizioni del modello e ridurne l'accuratezza.

Tuttavia, nonostante queste difficoltà, gli LLM si sono
dimostrati molto efficaci in una varietà di task di NLP
(Natural Language Processing) e promettono risultati
soddisfacenti per il problema della rilevazione di spoiler
in un testo.

I dettagli dell'approccio proposto verranno discussi nei
seguenti capitoli.
