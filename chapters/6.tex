\chapter{Considerazioni Finali}
\label{ch:owari}

In conclusione, i risultati ottenuti dimostrano che il
fine-tuning e la scelta del modello giusto sono
fondamentali per ottenere prestazioni ottimali nel
rilevamento degli spoiler, ma anche quanto sia difficile
valutare la qualità dei risultati ottenuti.

L'architettura del sistema si è dimostrata
sorprendentemente robusta e in grado di fornire risultati
di qualità migliore rispetto a quelli ottenuti con modelli
più complessi e costosi in termini di risorse
computazionali.
Questo è un aspetto importante, poiché dimostra che è
possibile ottenere risultati di alta qualità senza
dipendere da servizi cloud costosi e senza dover affrontare
i problemi di privacy e sicurezza associati all'uso di tali
servizi.

L'addestramento dei modelli di embedding mostrano che è
possibile utilizzare LLM pre-addestrati senza conoscenze
specifiche del dominio per ottenere risultati competitivi.
In caso di domini specifici, è possibile ottenere risultati
migliori allenando un modello di embedding specifico sul
dominio, cosa che rende l'approccio più flessibile rispetto
ad allenare un modello di linguaggio da zero.
Questo dimostra nuovamente che la scelta del modello giusto
è fondamentale per ottenere prestazioni ottimali.
Tuttavia, è importante notare che la latenza del sistema è
un fattore cruciale da considerare, poiché il sistema è
progettato per essere utilizzata in tempo reale da utenti
finali.

\section{Lavori Futuri}
\label{sec:future-work}

Lo scopo di questo lavoro è stato quello di dimostrare un
approccio alternativo al rilevamento degli spoiler
utilizzando modelli generativi avanzati.
Questo lavoro è stato quanto di più ambizioso si potesse
fare in un tempo limitato e con le risorse disponibili, ma
ha comunque portato a risultati interessante e promettenti.
Tuttavia, ci sono ancora molte aree in cui è possibile
migliorare ed espandere questo lavoro.

In primo luogo, si è scelto di non ottimizzare al meglio
nessuna delle parti del sistema, ma di concentrarsi
sull'architettura del sistema in generale.
In futuro, sarebbe interessante ottimizzare ulteriormente
le prestazioni del sistema, sperimentando con diverse
architetture e tecniche di ottimizzazione.

Lavorare a un sistema di rilevamento degli spoiler
completamente automatizzato è un obiettivo ambizioso, ma è
possibile farlo utilizzando tecnologie di disponibili al
pubblico e open source.

Lascio quindi a chiunque voglia continuare questo lavoro
l'idea di esplorare le possibilità offerte dai modelli
generativi avanzati con l'obiettivo di migliorare
l'esperienza dell'utente finale.